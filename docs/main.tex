\documentclass{article}

\usepackage{amsmath}
\usepackage{amssymb}
\usepackage{amsfonts}
\usepackage{framed}
\usepackage{hyperref}

\usepackage{fontspec}
	\setmainfont{EB Garamond}

\usepackage{polyglossia}
	\setmainlanguage{english}

\usepackage{biblatex}

\usepackage{tikz}
	\usetikzlibrary{graphs}
	\usetikzlibrary{quotes}

\newcommand \rewriterule [3] {
	\begin{samepage}
	\begin{framed}
	\label{#1}
	\textsf{#1}
	\begin{displaymath}
		\vcenter{
			\hbox{
				\begin{tikzpicture}
					\graph[grow right=50pt] { #2 } ;
				\end{tikzpicture}
			}
		} \implies \vcenter{
			\hbox{
				\begin{tikzpicture}
					\graph[grow right=50pt] { #3 } ;
				\end{tikzpicture}
			}
		}
	\end{displaymath}
	\end{framed}
	\end{samepage}
}

\title{Graph rewritings}
\author{}
\date{}

\begin{document}

    \maketitle

	\section{Rules}

	This are the rewrite rules presented in
	\textcite{SchusterEnhanced2016} for the Enhanced UD
	representations. They are represented as patterns in graphs,
	somewhat informally. Later, we'll discuss what it takes to
	formalize them.

	\rewriterule {augmented-modifiers} {
		a ->["\texttt{nmod}"] b ->["\texttt{case}"] c
	} {
		a ->["\texttt{nmod:}c"] b ->["\texttt{case}"] c
	}

	\rewriterule {augmented-conjuncts} {
		a ->["\texttt{cc}"] b ;
		a ->["\texttt{conj}"'] c ;
	} {
		a ->["\texttt{cc}"] b ;
		a ->["\texttt{conj:}b"'] c ;
	}

	\textbf{Note:} this one has an additional condition; that if
	two \texttt{cc} relations come out of the same node, the one
	that's used is the one whose target precedes $\mathrm{c}$, or
	the first one to appear if none do (cf. the source of CoreNLPP, at
	\texttt{src/edu/stanford/nlp/trees/UniversalEnglishGrammaticalStructure.java:720}
	in the commit \texttt{3499d27e615c35702f23948e886a7389b5695c33}:
	``In case multiple coordination marker[s] depend on the same
	governor[,] the one that precedes the conjunct is appended to the
	conjunction relation or the first one if no preceding marker
	exists.''). Importantly, this means \emph{sentence word order is
	still relevant at this stage}. So we should consider a graph for
	that order as well, and our language will have to be enhanced
	to account for such conditions.

	\rewriterule {propagate-governors-and-dependents} {
		h ->["r $\in$ (\texttt{nsubj}|\texttt{amod})"] a ;
		a ->["\texttt{conj:}x"] b ;
	} {
		{
			h ->["r"] {a , b} ;
			a ->["\texttt{conj:}x"] b
		}
	}

	(Not complete; case for verbs is missing)

	\rewriterule {subjects-of-controlled-verbs} {
		s <-["\texttt{nsubj}"] v ->["\texttt{xcomp}"] c
	} {
		{
			s <-["\texttt{nsubj}"] v ->["\texttt{xcomp}"] c ,
			s <-["\texttt{nsubj:xsubj}"', bend right] c
		}
	}

	\section{Formalization}

	The graph rewriting rules presented by Schuster are apparently just pairs of graphs. But there's more to it. For one, there's a matter of binding: variables in the right-hand side (RHS) of the rules are \emph{references}, whereas the ones in the left-hand side (LHS) are \emph{introductions}. That is, the intended semantics of the rule requires us to pattern-match the LHS to subgraphs of an argument sentence, and then replace it with the RHS with the corresponding variables replaced by the values bound to them by the matching.
	
	Besides, there's the matter of additional conditions not expressible, apparently, as graph matching. Look at \hyperref[augmented-conjuncts]{\textsf{augmented-conjuncts}}; there, there's a restriction so that the pattern can only match the closest node connected to $a$ via $\texttt{cc}$ is the one that's used.

% {}=gov  [>conj ({}=conj >cc {}=cc) |  >conj ({} >cc {}=cc) >conj {}=conj ]

%      conj         cc
% gov ------> conj ----> cc
%
%     or
%
%      conj      cc
% gov ------> _ ----> cc
%  \
%   `------> conj
%     conj
%


\end{document}
